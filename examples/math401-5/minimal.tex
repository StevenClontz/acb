%**************************************%
%*    Generated from PreTeXt source   *%
%*    on 2017-11-01T14:40:42-04:00    *%
%*                                    *%
%*   http://mathbook.pugetsound.edu   *%
%*                                    *%
%**************************************%
\documentclass[10pt,]{article}
%% Custom Preamble Entries, early (use latex.preamble.early)
%% Inline math delimiters, \(, \), need to be robust
%% 2016-01-31:  latexrelease.sty  supersedes  fixltx2e.sty
%% If  latexrelease.sty  exists, bugfix is in kernel
%% If not, bugfix is in  fixltx2e.sty
%% See:  https://tug.org/TUGboat/tb36-3/tb114ltnews22.pdf
%% and read "Fewer fragile commands" in distribution's  latexchanges.pdf
\IfFileExists{latexrelease.sty}{}{\usepackage{fixltx2e}}
%% Text height identically 9 inches, text width varies on point size
%% See Bringhurst 2.1.1 on measure for recommendations
%% 75 characters per line (count spaces, punctuation) is target
%% which is the upper limit of Bringhurst's recommendations
%% Load geometry package to allow page margin adjustments
\usepackage{geometry}
\geometry{letterpaper,total={340pt,9.0in}}
%% Custom Page Layout Adjustments (use latex.geometry)
%% This LaTeX file may be compiled with pdflatex, xelatex, or lualatex
%% The following provides engine-specific capabilities
%% Generally, xelatex and lualatex will do better languages other than US English
%% You can pick from the conditional if you will only ever use one engine
\usepackage{ifthen}
\usepackage{ifxetex,ifluatex}
\ifthenelse{\boolean{xetex} \or \boolean{luatex}}{%
%% begin: xelatex and lualatex-specific configuration
%% fontspec package will make Latin Modern (lmodern) the default font
\ifxetex\usepackage{xltxtra}\fi
\usepackage{fontspec}
%% realscripts is the only part of xltxtra relevant to lualatex 
\ifluatex\usepackage{realscripts}\fi
%% 
%% Extensive support for other languages
\usepackage{polyglossia}
%% Main document language is US English
\setdefaultlanguage{english}
%% Spanish
\setotherlanguage{spanish}
%% Vietnamese
\setotherlanguage{vietnamese}
%% end: xelatex and lualatex-specific configuration
}{%
%% begin: pdflatex-specific configuration
%% translate common Unicode to their LaTeX equivalents
%% Also, fontenc with T1 makes CM-Super the default font
%% (\input{ix-utf8enc.dfu} from the "inputenx" package is possible addition (broken?)
\usepackage[T1]{fontenc}
\usepackage[utf8]{inputenc}
%% end: pdflatex-specific configuration
}
%% Symbols, align environment, bracket-matrix
\usepackage{amsmath}
\usepackage{amssymb}
%% allow page breaks within display mathematics anywhere
%% level 4 is maximally permissive
%% this is exactly the opposite of AMSmath package philosophy
%% there are per-display, and per-equation options to control this
%% split, aligned, gathered, and alignedat are not affected
\allowdisplaybreaks[4]
%% allow more columns to a matrix
%% can make this even bigger by overriding with  latex.preamble.late  processing option
\setcounter{MaxMatrixCols}{30}
%%
%% Color support, xcolor package
%% Always loaded.  Used for:
%% mdframed boxes, add/delete text, author tools
\PassOptionsToPackage{usenames,dvipsnames,svgnames,table}{xcolor}
\usepackage{xcolor}
%%
%% Semantic Macros
%% To preserve meaning in a LaTeX file
%% Only defined here if required in this document
%% Subdivision Numbering, Chapters, Sections, Subsections, etc
%% Subdivision numbers may be turned off at some level ("depth")
%% A section *always* has depth 1, contrary to us counting from the document root
%% The latex default is 3.  If a larger number is present here, then
%% removing this command may make some cross-references ambiguous
%% The precursor variable $numbering-maxlevel is checked for consistency in the common XSL file
\setcounter{secnumdepth}{3}
%% Environments with amsthm package
%% Theorem-like environments in "plain" style, with or without proof
\usepackage{amsthm}
\theoremstyle{plain}
%% Numbering for Theorems, Conjectures, Examples, Figures, etc
%% Controlled by  numbering.theorems.level  processing parameter
%% Always need a theorem environment to set base numbering scheme
%% even if document has no theorems (but has other environments)
\newtheorem{theorem}{Theorem}[section]
%% Only variants actually used in document appear here
%% Style is like a theorem, and for statements without proofs
%% Numbering: all theorem-like numbered consecutively
%% i.e. Corollary 4.3 follows Theorem 4.2
%% Definition-like environments, normal text
%% Numbering is in sync with theorems, etc
\theoremstyle{definition}
\newtheorem{definition}[theorem]{Definition}
%% Localize LaTeX supplied names (possibly none)
\renewcommand*{\proofname}{Proof}
\renewcommand*{\abstractname}{Abstract}
%% Equation Numbering
%% Controlled by  numbering.equations.level  processing parameter
\numberwithin{equation}{section}
%% Raster graphics inclusion, wrapped figures in paragraphs
%% \resizebox sometimes used for images in side-by-side layout
\usepackage{graphicx}
%%
%% More flexible list management, esp. for references and exercises
%% But also for specifying labels (i.e. custom order) on nested lists
\usepackage{enumitem}
%% hyperref driver does not need to be specified, it will be detected
\usepackage{hyperref}
%% Hyperlinking active in PDFs, all links solid and blue
\hypersetup{colorlinks=true,linkcolor=blue,citecolor=blue,filecolor=blue,urlcolor=blue}
\hypersetup{pdftitle={Properties of Convergent Sequences}}
%% If you manually remove hyperref, leave in this next command
\providecommand\phantomsection{}
%% If tikz has been loaded, replace ampersand with \amp macro
%% extpfeil package for certain extensible arrows,
%% as also provided by MathJax extension of the same name
%% NB: this package loads mtools, which loads calc, which redefines
%%     \setlength, so it can be removed if it seems to be in the 
%%     way and your math does not use:
%%     
%%     \xtwoheadrightarrow, \xtwoheadleftarrow, \xmapsto, \xlongequal, \xtofrom
%%     
%%     we have had to be extra careful with variable thickness
%%     lines in tables, and so also load this package late
\usepackage{extpfeil}
%% Custom Preamble Entries, late (use latex.preamble.late)
%% Begin: Author-provided packages
%% (From  docinfo/latex-preamble/package  elements)
%% End: Author-provided packages
%% Begin: Author-provided macros
%% (From  docinfo/macros  element)
%% Plus three from MBX for XML characters
\newcommand{\doubler}[1]{2#1}
\newcommand{\lt}{<}
\newcommand{\gt}{>}
\newcommand{\amp}{&}
%% End: Author-provided macros
%% Title page information for article
\title{Properties of Convergent Sequences}
\author{Matt Salomone\\
Bridgewater State University
}
\date{November 1, 2017}
\begin{document}
%% Target for xref to top-level element is document start
\hypertarget{minimal}{}
\maketitle
\thispagestyle{empty}
\begin{abstract}
\hypertarget{p-1}{}%
Convergence is a difficult definition to use directly. We investigate how other, more easily verifiable properties of sequences of real numbers are related to convergence.%
\end{abstract}
\hypertarget{p-2}{}%
This is a short paragraph to introduce the article (but it is not the abstract).  It is optional, in case it would be preferable to have the first section be titled an ``Introduction.''%
\typeout{************************************************}
\typeout{Section 1 Bounded Sequences}
\typeout{************************************************}
\section[{Bounded Sequences}]{Bounded Sequences}\label{section-bounded}
\hypertarget{p-3}{}%
Convergence is a very strong property for a sequence to have, since it requires the tails of the sequence to all grow arbitrarily close to a specified real number (its limit). Let's look at some simpler properties, each of which is weaker than convergence, and their relationships to convergence.%
\begin{definition}[{Bounded sequence}]\label{definition-1}
\(s_n\)\emph{bounded}\(M \in\mathbb{R}\)\(n \in\mathbb{N}\)%
\begin{equation*}
|s_n| \leq M.
\end{equation*}
\end{definition}
\hypertarget{p-4}{}%
Intuitively, we might say that \emph{all} terms of a bounded sequence lie between a constant "ceiling" and a constant "floor:" another way to write the inequality at the end of the definition is \(-M \leq |s_n| \leq M.\)%
\par
\hypertarget{p-5}{}%
Boundedness is indeed a weaker condition than convergence; while it is not true that every bounded sequence is convergent, is \emph{is} true that every convergent sequence is bounded:%
\begin{theorem}[{Convergent implies bounded}]\label{theorem-1}
\(s_n\)\(s_n\)\end{theorem}
\begin{proof}\hypertarget{proof-1}{}
\hypertarget{p-6}{}%
Intuitively, convergence is a strong condition. Given any \(\epsilon>0\), it produces an \(N\in\mathbb{N}\) which divides the sequence into a (finite) head and an (infinite) tail. We imagine that each will have a ceiling of its own:%
\begin{tabular}{m{.2\linewidth}m{.6\linewidth}}
\includegraphics[width=\linewidth]{images/video-1.jpg}&%
\href{https://www.youtube.com/watch?v=uWjC7e8Rh_Q}{\texttt{\nolinkurl{www.youtube.com/watch?v=uWjC7e8Rh_Q}}}
\end{tabular}
\hypertarget{p-7}{}%
Now, let us define \(\epsilon = 1\). Since \(s_n\) is a convergent sequence, let us denote \(L = \lim_{n\to\infty} s_n.\)%
\par
\hypertarget{p-8}{}%
Then by definition of convergence there exists an \(N\in\mathbb{N}\) such that, for all \(n \geq N\), we have%
\begin{equation*}
|s_n - L| \lt \epsilon.
\end{equation*}
This defines for us a head of the sequence, \(\{s_1,s_2,\ldots,s_N\}\), and a tail of the sequence, \(\{s_N,s_{N+1},s_{N+2},\ldots\}\), and all of the terms in the \emph{tail} are within a distance of \(\epsilon\) of the limit \(L\).%
\par
\hypertarget{p-9}{}%
Using an add-subtract trick can shift the inequality \(|s_n-L|\lt\epsilon\) from a measurement of the sequence's distance from \(L\) into a measurement of its distance from zero (i.e., its absolute value):%
\begin{align}
|s_n|  = |s_n - L + L| \leq |s_n-L| + |L|  \lt \epsilon + |L| \amp\amp \text{for all } n \geq N. \label{tail-ineq}
\end{align}
In other words, \(\epsilon+|L|\) is an upper bound for the tail of the sequence.%
\par
\hypertarget{p-10}{}%
Meanwhile, since the head of the sequence is a \emph{finite} set, it will in particular have a largest element that can be used as an upper bound for that set. So we define \(m = \max \{ |s_1|, |s_2|, \ldots, |s_N| \}.\)%
\par
\hypertarget{p-11}{}%
Now define \(M = \max\{ m , \epsilon+|L|\}.\)%
\par
\hypertarget{p-12}{}%
Let \(n\in\mathbb{N}\) be arbitrarily chosen. Then there are two cases, depending on whether the \(n\)th term belongs to the head of the sequence or the tail:%
\leavevmode%
\begin{enumerate}
\item\hypertarget{li-1}{}\emph{If \(n \leq N\)}, then \(s_n\) belongs to the head of the sequence and \(|s_n|\) is one of the values in the finite list which was used to define the maximum of the head, \(m\). Hence \(|s_n|\leq m \leq M.\)%
\item\hypertarget{li-2}{}\emph{If \(n \gt N\)}, then \(s_n\) belongs to the tail of the sequence and \(|s_n|\) is governed by the tail inequality \hyperref[tail-ineq]{(\ref{tail-ineq})}. Hence \(|s_n| \lt \epsilon+|L| \leq M.\)%
\end{enumerate}
\hypertarget{p-13}{}%
This covers all cases, so we conclude that for all \(n\in\mathbb{N}\) we have \(|s_n|\leq M.\)%
\begin{tabular}{m{.2\linewidth}m{.6\linewidth}}
\includegraphics[width=\linewidth]{images/video-2.jpg}&%
\href{https://www.youtube.com/watch?v=DpfmgXilu_8}{\texttt{\nolinkurl{www.youtube.com/watch?v=DpfmgXilu_8}}}
\end{tabular}
\end{proof}
\typeout{************************************************}
\typeout{Section 2 Monotonic Sequences}
\typeout{************************************************}
\section[{Monotonic Sequences}]{Monotonic Sequences}\label{section-monotonic}
\hypertarget{p-14}{}%
Another class of sequences whose behavior is well regulated is the class of sequences which "do not change direction." These are the monotonic sequences.%
\begin{definition}[{Monotonic sequence}]\label{definition-2}
\(s_n\)\emph{monotonic}\leavevmode%
\begin{itemize}[label=\textbullet]
\item{}For all \(n\in\mathbb{N}\), we have \(s_n \leq s_{n+1}.\)%
\item{}For all \(n\in\mathbb{N}\), we have \(s_n \geq s_{n+1}.\)%
\end{itemize}
In the first case, we say the sequence is \emph{increasing}. In the second case, we say the sequence is \emph{decreasing.} If either inequality is a strict inequality (\(\lt\) or \(\gt\)), then we say the sequence is "strictly" increasing or decreasing respectively.\end{definition}
\hypertarget{p-15}{}%
%
\typeout{************************************************}
\typeout{Section 3 Cauchy Sequences}
\typeout{************************************************}
\section[{Cauchy Sequences}]{Cauchy Sequences}\label{section-cauchy}
\hypertarget{p-16}{}%
Hi there.%
\end{document}